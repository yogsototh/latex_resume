%------------------------------------
% Paquet contenant les options de présentations
% a savoir :
%   Layout, couleurs, fontes
%
% Yann Esposito inpired by
% Dario Taraborelli
% Typesetting your academic CV in LaTeX
%
% URL: http://nitens.org/taraborelli/cvtex
% DISCLAIMER: This template is provided for free and without any guarantee 
% that it will correctly compile on your system if you have a non-standard  
% configuration.
%------------------------------------

%!TEX TS-program = xelatex
%!TEX encoding = UTF-8 Unicode

\documentclass[12pt, a4paper]{article}
\usepackage[french]{babel} % pour le français
\usepackage{fontspec} % pour les fontes
\usepackage{marvosym} % pour les symboles de téléphone, de lettre et de fax
\usepackage{xltxtra}
\usepackage{graphicx}

%% pour le problème du nombre d'entrées trop grande
%% utiliser la commande \FloatBarrier dans les blocs
%% trops longs et décomenter l'include suivant
% \usepackage{placeins}

% DOCUMENT LAYOUT
\usepackage{geometry} 
\geometry{a4paper, textwidth=5.5in, textheight=8.5in, marginparsep=7pt, marginparwidth=.6in}
\setlength\parindent{0in}

% FONTS
\defaultfontfeatures{Mapping=tex-text} % converts LaTeX specials (``quotes'' --- dashes etc.) to unicode
\newcommand{\myRomanFont}{Hoefler Text}
\newcommand{\myMonoFont}{Monaco}
\newcommand{\mySansFont}{Optima}
\setromanfont[Ligatures={Common,Rare},Numbers=Lining]{\myRomanFont}
\setmonofont[Scale=0.8]{\myMonoFont} 
\setsansfont[Scale=0.9]{\mySansFont} 
% ---- CUSTOM AMPERSAND
\newcommand{\amper}{{\fontspec[Scale=.95]{Hoefler Text}\selectfont\itshape\&}}

% PIED DE PAGE
\usepackage{fancyhdr,lastpage}
\pagestyle{fancy}
\renewcommand{\headrulewidth}{0pt}
\renewcommand{\footrulewidth}{0.4pt}
\lfoot{{\scriptsize mis à jour le \today}}
\cfoot{ \vspace{-10pt} {\setFootFont \fontsize{8pt}{10pt} {\va Y}an{\vf n} {\va E}{\vb s}{\vb p}o{\vg s}{\vb i}{\va t}{\vd o}} }
\rfoot{\arabic{page}}

% HEADINGS
\usepackage{sectsty} 
\usepackage[normalem]{ulem} 
\sectionfont{\sffamily\mdseries\large\underline} 
\subsectionfont{\rmfamily\mdseries\scshape\normalsize} 
\subsubsectionfont{\rmfamily\bfseries\upshape\normalsize} 

% PDF SETUP
% ---- FILL IN HERE THE DOC TITLE AND AUTHOR
\usepackage[dvipdfm, bookmarks, colorlinks, breaklinks, pdftitle={Yann Esposito - CV},pdfauthor={Yann Esposito}]{hyperref}  

% Mes couleurs persos (je n'utilise plus vraiment l'orange en fait)
% Il faudrait utiliser #600 #d00 pour les rouges, 
% #ddd et #666 pour les gris.
\usepackage{color}
\definecolor{orange}{rgb}{1,0.5,0}
\definecolor{darkorange}{rgb}{0.5,0.25,0}
\definecolor{red}{rgb}{0.6875,0,0}
\definecolor{darkred}{rgb}{0.375,0,0}
\hypersetup{linkcolor=red,citecolor=red,filecolor=black,urlcolor=red} 


% Yann Esposito
%
% Les commanes LaTeX pour les CV
%
% Des macros pour afficher correctement les contacts
% Les publications (en mode 'full' ou 'short')
%


% années dans la marge
\newcommand{\years}[1]{\marginpar{\scriptsize #1}}

% ======
% AUTEUR
% ======
%
% Simple : \auteur{François Dupond}
% Expert : \auteur{\va{F}ran\vb{ç}ois \vg{D}upond}
%
% les macros \va, \vb, ... \vg
% permette d'utiliser les variante des fontes
%
% Tout d'abord on sélectionne la fonte par défaut
% pour l'auteur ici Zapfino (très jolie script)
\newcommand{\setAuthorFont}[1]{
\color{darkred} 
\fontspec[
    Ligatures={Common,Rare},
    Variant=3
    ]
    {Zapfino}\fontsize{22pt}{25pt}\selectfont #1}%
\newcommand{\setFootFont}[1]{
\color{darkred} 
\fontspec[
    Ligatures={Common,Rare},
    Variant=3
    ]
    {Zapfino}\fontsize{10pt}{15pt}\selectfont #1}%
% Zapfino possède de multiples variantes pour ses lettres
% C'est pourquoi on en profite, une commande pour chaque
% Variante. Pour modifier une lettre dans son nom
% il suffit d'écrire \auteur{\vc{D}upond} par exemple
\newcommand{\va}[1]{\fontspec[Variant=1]{Zapfino}#1}
\newcommand{\vb}[1]{\fontspec[Variant=2]{Zapfino}#1}
\newcommand{\vc}[1]{\fontspec[Variant=3]{Zapfino}#1}
\newcommand{\vd}[1]{\fontspec[Variant=4]{Zapfino}#1}
\newcommand{\ve}[1]{\fontspec[Variant=5]{Zapfino}#1}
\newcommand{\vf}[1]{\fontspec[Variant=6]{Zapfino}#1}
\newcommand{\vg}[1]{\fontspec[Variant=7]{Zapfino}#1}
% La macro d'ateur proprement dite
\newcommand{\auteur}[1]{{\hspace{-2em}\vspace{-1em} \setAuthorFont{#1}}}

% =====
% TITRE
% =====
%
% Simple : \titre{Chercheur en informatique}
%
\newcommand{\setTitreFont}[1]{
\fontspec[]{Optima}\fontsize{18pt}{23pt}\selectfont #1}%
% La macro d'ateur proprement dite
\newcommand{\titre}[1]{\begin{minipage}[t]{.5\linewidth}\vspace{.5em}\flushright{\setTitreFont{#1}}\end{minipage}\\[1cm]}

% =======
% CONTACT
% =======
%
% exemple :
% ---
%
% \rue{18 rue des Anciens}
% \codePostal{012345}
% \ville{Arkkam}
% \telephone{(+33)1 23 45 67 89}
% \email{lovecraft@cthulhu.com}
%
% \afficheContactInfos
% ---
%
% batiment, rue, complementAdresse, codePostal, ville, pays
% telephone, site, email, faxnum
\newcommand{\batiment}[1]{\newcommand{\batimentValue}{#1}} % batiment
\newcommand{\rue}[1]{\newcommand{\rueValue}{#1}} % rue
\newcommand{\complementAdresse}[1]{\newcommand{\complementAdresseValue}{#1}} % complement
\newcommand{\codePostal}[1]{\newcommand{\codePostalValue}{#1}} % code postal
\newcommand{\ville}[1]{\newcommand{\villeValue}{#1}} % ville
\newcommand{\pays}[1]{\newcommand{\paysValue}{#1}} % pays
\newcommand{\telephone}[1]{\newcommand{\telephoneValue}{#1}} % téléphone
\newcommand{\site}[1]{\newcommand{\siteValue}{#1}} % site
\newcommand{\email}[1]{\newcommand{\emailValue}{#1}} % email
\newcommand{\faxnum}[1]{\newcommand{\faxValue}{#1}} % fax

% affichage de l'adresse
\newcommand{\addressPrefix}{\Letter\hspace{.8em}}
\newcommand{\afficheAdresse}{
    \addressPrefix
    \ifdefined\batimentValue
        Bât. \batimentValue, 
    \else\fi
        \rueValue\\
    \ifdefined\complementAdresseValue
        \phantom{\addressPrefix}\complementAdresseValue \\
    \else\fi
        \phantom{\addressPrefix}\texttt{\codePostalValue}, \villeValue\ifdefined\paysValue
    \\
        \phantom{\addressPrefix}\paysValue
    \else\fi
    \\[.2cm]
}

% Affiche l'adresse ainsi que les informations de contacts
\newcommand{\afficheContactInfos}{
    \begin{minipage}[t]{0.5\linewidth}
        \afficheAdresse
    \end{minipage}
    \begin{minipage}[t]{0.5\linewidth}
        \begin{flushright}
        \ifdefined\telephoneValue
            \Telefon\hspace{1em} {\texttt \telephoneValue}
        \else\fi
        \ifdefined\faxValue
            \\
            \fax\hspace{1em} {\texttt \faxValue}
        \else\fi
        \ifdefined\emailValue
            \href{mailto:\emailValue}{\texttt \emailValue}
        \else\fi
        \ifdefined\siteValue
            \\
            \href{\siteValue}{ \texttt \siteValue }
        \else\fi
        \end{flushright}
    \end{minipage}
}

% ============
% PUBLICATIONS
% ============
% Les macros a utiliser :
% \article et \journalentry qui prennent 5 paramètres :
%   année
%   titre
%   auteur(s)
%   conférence ou journal
%   détails
%
% Pour afficher les publications en mode short
% Il suffit de décommenter la ligne suivante
% \newcommand{\short}{Short publications}


% Commandes pour les publications
\newcommand{\articleTitle}[1]{\emph{#1}}
\newcommand{\conf}[1]{{\textcolor{red}{#1}}}
\newcommand{\journal}[1]{\textcolor{red}{#1}}
\newcommand{\details}[1]{{\footnotesize #1}}

% paramètres:
%   année
%   titre
%   auteur(s)
%   conférence ou journal
%   détails
\newcommand{\article}[5]{
    \ifdefined\short
        \shortarticle{#1}{#2}{#3}{#4}{#5}
    \else
        \longarticle{#1}{#2}{#3}{#4}{#5}
    \fi
}
\newcommand{\journalentry}[5]{
    \ifdefined\short
        \shortjournalentry{#1}{#2}{#3}{#4}{#5}
    \else
        \longjournalentry{#1}{#2}{#3}{#4}{#5}
    \fi
}

\newcommand{\longarticle}[5]{
    \noindent\years{#1}\begin{minipage}[t]{\linewidth}\articleTitle{#2},\\#3\\\conf{#4}, \details{#5}.\end{minipage}\smallskip\\
}

\newcommand{\longjournalentry}[5]{
    \noindent\years{#1}\begin{minipage}[t]{\linewidth}\articleTitle{#2},\\#3\\\journal{#4}, \details{#5}.\end{minipage}\smallskip\\
}

\newcommand{\shortarticle}[5]{
    [#4]
}
\newcommand{\shortjournalentry}[5]{
    [#4~#1]
}

% ============
% PRESENTATION
% ============
% une macro \deuxcolones qui permet de simuler
% la presentation un peu comme avec un tableau
%
%   \deuxcolones{Total}{370 heures}
%
% on peut changer la taille des colones avec :
% \renewcommand{\dcleftwidth}{.15\linewidth}
% \renewcommand{\dcrightwidth}{.85\linewidth}
%
\newcommand{\dcleftwidth}{.25\linewidth}
\newcommand{\dcrightwidth}{.75\linewidth}

\newcommand{\deuxcolones}[2]{
\begin{minipage}[t]{\dcleftwidth}
#1
\end{minipage}
\begin{minipage}[t]{\dcrightwidth}
\small
#2
\end{minipage}
\medskip
}

\begin{document}
\reversemarginpar

% --- Contact ---
\batiment{K4}
\rue{101, rue Jean Giono}
\complementAdresse{Les Pugets}
\codePostal{06700}
\ville{Saint-Laurent-du-Var}
\pays{France}

\telephone{(+33)6 50 84 52 71}
% fax{}
\email{yann.esposito@gmail.com}
\site{http://yannesposito.com}

% --- L'affichage commence ---
\auteur{{\va Y}an{\vf n} {\va E}{\vb s}{\vb p}o{\vg s}{\vb i}{\va t}{\vd o}}
\titre{Expert en Data Mining}
\afficheContactInfos

% \hrule
\section*{Expérience professionnelle}

\noindent\years{depuis 2007}\emph{Consultant}, Air France via la société de service Astek, Sophia Antipolis.\\
\years{2006-2007}\emph{Post Doctorant}, Laboratoire Hubert Curien, Saint-Etienne.\\
\years{2004-2005}\emph{\bsc{ater} {\footnotesize (enseignement et recherche)}}, Université de Provence, Marseille.\\
\years{2001-2004}\emph{Moniteur (enseignant \amper{} chercheur)}, Université de Provence, Marseille.\\
\years{1999-2000}\emph{Vacataire}, Université Aix-Marseille III, Marseille.\\
\years{avant 1999}\emph{Divers emplois d'été}, {\footnotesize (Eurocopter, Naphtachimie, LCL...) } aux alentours de Marseille.

\section*{Applications scientifiques développées}
% SEDiL Stochastic Edit Distance Learning (avec la coloration et les smallcaps
\textsc{sed}\textit{\footnotesize i}\textsc{l} ({\textsc{\color{darkred}s}tochastic 
\textsc{\color{darkred}e}dit
\textsc{\color{darkred}d}{\color{darkred}i}stance
\textsc{\color{darkred}l}earning}), 
inférence de matrices de similarité.\\
{\footnotesize \phantom{espace}Public: biologistes}\\
{\footnotesize \phantom{espace}Site : \href{http://yann.esposito.free.fr/sedil.php}{http://yann.esposito.free.fr/sedil.php}}

%DEES
\textsc{dees}, inférence d'automates probabilistes.\\
{\footnotesize \phantom{espace}Public: informaticiens}\\
{\footnotesize \phantom{espace}Site : \href{http://yann.esposito.free.fr/dees.php}{http://yann.esposito.free.fr/dees.php}}

%\hrule
\section*{Diplômes}

\noindent\years{2004}\emph{Doctorat en informatique}, Université de Provence, Marseille.\\
\noindent\years{2001}\emph{\bsc{dea} d'informatique} , Université de la Méditerranée, Marseille.\\
\noindent\years{2000}\emph{Maîtrise d'Informatique {\footnotesize(mention bien)}}, Université de Provence, Marseille.\\
\noindent\years{1999}\emph{Licence d'Informatique {\footnotesize(mention bien)}}, Université de Provence, Marseille.\\
\noindent\years{1998}\emph{\bsc{deug mias} {\footnotesize option mathématiques}}, Université de Provence, Marseille.\\
\noindent\years{1995}\emph{\bsc{bac s} {\footnotesize option mathématiques}}, Lycée P. Langevin, Martigues.

%\hrule
\section*{Projets}

\subsection*{Université}

\noindent\years{2006}Conception d'un protocole réseau anonyme pour l'enseignement des Réseaux\\
\years{2003-2004}\bsc{aci} "Masses de Données" \texttt{Genoto3d}\\
{\phantom{es}\footnotesize Apprentissage automatique appliqué à la prédiction de la structure tertiaire des protéines}\\
\years{2003}Mise en place d'un site sous PHP/MySQL de gestion de conférence scientifque

\subsection*{Projets diffusés sur Internet}

\noindent\years{2007-2009}\emph{YPassword}, widget de gestion de mot de passes\\
\noindent\years{2007}\emph{YClock}, économiseur d'écran pour Mac OS X\\
\noindent\years{2003-2004}Conception d'un package metapost permettant l'affichage d'automates probabilistes\\
\noindent\years{2003-2004}\emph{YAquaBubbles}, économiseur d'écran pour Mac OS X

\section*{Compétences informatiques}

\deuxcolones{Langages objets}{C++, Java, Python, Perl, Objective-C}
\deuxcolones{\bsc{api}}{
    C++{\footnotesize (\bsc{stl})}, Java {\footnotesize (Swing/Java 2D)},
    Python {\footnotesize (PySQLite, wxPython, crypto, socket)},
    Objective-C {\footnotesize (Cocoa, Core Data, Quartz Composer)}}
\deuxcolones{Autres langages}{C, camL}
\deuxcolones{Outils Unix}{bash, zsh, awk, \LaTeX, \XeLaTeX, ConTeXt, Metapost}
\deuxcolones{Méthodes}{\bsc{uml}, Merise}
\deuxcolones{\bsc{ide}}{XCode, Environnement Unix}
\deuxcolones{Outils décisionnels}{Entreprise Miner (SAS), weka, $\textrm{\bsc{svm}}_\textrm{\scriptsize{\textit{light}}}$}
\deuxcolones{Normes}{\bsc{html}, \bsc{xhtml}, \bsc{css}, \bsc{svg}}
\deuxcolones{contrôle version}{\bsc{git}, \bsc{bzr}, subversion, \bsc{cvs}}
\deuxcolones{Environnements}{Windows, Unix/Linux (Solaris, Ubuntu, Debian), Mac OS X}
\deuxcolones{\bsc{ihm}}{Java Swing, wxPython, Cocoa}
\deuxcolones{Protocoles réseaux}{Ethernet, TCP/IP, UDP, FTP, DNS, DHCP}
\deuxcolones{Outils réseaux}{ifconfig, tcpdump, wireshark (ehtereal),lsof...}
\deuxcolones{Sécurité}{\bsc{aes}, \bsc{des}, \bsc{3des},Codes de Hamming, \bsc{pgp}, \bsc{gpg}, réseaux anonymes...}
\deuxcolones{Bases de données}{MySQL, PostgreSQL, SQLite}

\section*{Publications \amper{} présentations}

\subsection*{Articles de Journaux internationnaux}

\journalentry{2008}{On Rational Stochastic Languages}
{F. Denis, Y. Esposito,}{Fundamenta~Informatic\ae{}}
{vol. 86, (1-2), p. 41-77, 2008}\\
\journalentry{2005}
{Links between Probabilistic Automata and Hidden Markov Models: probability distributions}
{ P. Dupont, F. Denis and Y. Esposito,}{Pattern Recognition}
{Special Issue on Grammatical Inference Techniques \amper{} Applications, Vol 38/9, pp 1349-1371, 2005}

\subsection*{Conférences internationnales}

\article{2008}{SEDiL: Software for Edit Distance Learning}
{Boyer L, Esposito Y , Habrard A , Oncina J , Sebban M}{ECML 2008}
{(accepted paper) Proceedings of the 19th European Conference on Machine Learning}



\article
{2006} % année
{Using Pseudo-Stochastic Rational Languages in Probabilistic Grammatical Inference} % titre
{A. Habrard, Francois Denis, and Y. Esposito} % auteurs
{ICGI 2006} % Conférence
{LNCS 4291, 112-124} % Détails de l'article
\article
{2006}
{Learning Rational stochastic languages}
{F. Denis, Y. Esposito, A. Habrard}
{COLT 2006}
{19th Annual Conference on Learning Theory, LNAI 4005, 274-288, 2006}
\article
{2004}
{Learning classes of Probabilistic Automata}
{F. Denis, Y. Esposito}
{COLT 2004}
{17th Annual Conference on Learning Theory, LNAI 3120, 124-139, 2004}
\article
{2003}
{Residual languages and probabilistic automata}
{F. Denis, Y. Esposito}
{ICALP}
{p. 452-463, LNCS 2719}
\article
{2002}
{Learning probabilistic residual finite state automata}
{Y. Esposito, Aurélien Lemay, F. Denis, Pierre Dupont}
{ICGI 2002}
{6th International Colloquium on Grammatical Inference,  LNAI 2484, 77-91, 2002}

\subsection*{Journaux nationaux}

\article
{2007}
{Identification à la limite d'automates probabilistes résiduels avec probabilité un}
{F. Denis, Y. Esposito}
{JEDAI}
{Vol. 6, Number 44, \href{http://jedai.afia-france.org/detail.php?PaperID=44}{http://jedai.afia-france.org/detail.php?PaperID=44}}

\subsection*{Conférences nationales}

\article
{2004}
{Identification in the limit of Probabilistic Non Deterministic Automata and Undecidable problem for Multiplicity Automata}
{F. Denis, Y. Esposito}
{CAP'04}
{Presses Universitaires de Grenoble, 81-96, 2004}
\article
{2003}
{Identification à la limite d'automates probabilistes avec probabilité de 1}
{F. Denis, Y. Esposito}
{CAP'03}
{Presses Universitaires de Grenoble, 249-264, 2003}

\subsection*{Articles publiés en ligne}

\article
{2006}
{Rational stochastic languages}
{ F. Denis, Y. Esposito,}
{Technical Report}
{\href{http://fr.arxiv.org/abs/cs.LG/0602093}{\texttt http\string://fr.arxiv.org/abs/cs.LG/0602093}}

\subsection*{Workshop}

\article
{2005}
{Rational Stochastic Languages}
{F. Denis, Y. Esposito,}
{TAGI 2005}
{Theoretical Aspect of Grammar Induction}

\subsection*{Thèse}
\article{2004}
{Contribution à l'inférence d'automates probabilistes}
{Y. Esposito}
{Université de Provence}
{Sous la direction de F. Denis}

\section*{Enseignements}

\deuxcolones{Total}
{\emph{370 heures}}
\deuxcolones{Matières}
{\emph{Fouille de données}, \emph{Apprentissage Automatique},\\
\emph{Logiciels Professionnels} (Gestion de projet),\\
\emph{Recherche opérationnelle}, \emph{Algorithmique},\\
\emph{Langage C}, \emph{Langage C++},\\
\emph{Systèmes}, \emph{Réseaux},\\
\emph{Initiation à l'informatique}.
}
\deuxcolones{Niveau}{\textsc{BAC+1 $\rightarrow$ BAC+5}}
\deuxcolones{Public}{DEUG {\scriptsize(math., physique, bio)}, Licence, Licence Pro,\\ Master, Master pro}

\footnotesize{
    Pour plus de détails voir mon ancien site 
    \href{http://www.lif.univ-mrs.fr/~esposito}
    {http://www.lif.univ-mrs.fr/\textasciitilde{}esposito}.
}


\end{document}
