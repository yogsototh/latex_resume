%------------------------------------
% Paquet contenant les options de présentations
% a savoir :
%   Layout, couleurs, fontes
%
% Yann Esposito inpired by
% Dario Taraborelli
% Typesetting your academic CV in LaTeX
%
% URL: http://nitens.org/taraborelli/cvtex
% DISCLAIMER: This template is provided for free and without any guarantee 
% that it will correctly compile on your system if you have a non-standard  
% configuration.
%------------------------------------

%!TEX TS-program = xelatex
%!TEX encoding = UTF-8 Unicode

\documentclass[12pt, a4paper]{article}
\usepackage{fontspec} % pour les fontes
\usepackage{marvosym} % pour les symboles de téléphone, de lettre et de fax
\usepackage{xltxtra}
\usepackage{graphicx}

%% pour le problème du nombre d'entrées trop grande
%% utiliser la commande \FloatBarrier dans les blocs
%% trops longs et décomenter l'include suivant
% \usepackage{placeins}

% DOCUMENT LAYOUT
\usepackage{geometry} 
\geometry{a4paper, textwidth=5.5in, textheight=8.5in, marginparsep=7pt, marginparwidth=.6in}
\setlength\parindent{0in}

% FONTS
\defaultfontfeatures{Mapping=tex-text} % converts LaTeX specials (``quotes'' --- dashes etc.) to unicode
\newcommand{\myRomanFont}{Hoefler Text}
\newcommand{\myMonoFont}{Monaco}
\newcommand{\mySansFont}{Optima}
\setromanfont[Ligatures={Common,Rare},Numbers=Lining]{\myRomanFont}
\setmonofont[Scale=0.8]{\myMonoFont} 
\setsansfont[Scale=0.9]{\mySansFont} 
% ---- CUSTOM AMPERSAND
\newcommand{\amper}{{\fontspec[Scale=.95]{Hoefler Text}\selectfont\itshape\&}}

% PIED DE PAGE
\usepackage{fancyhdr,lastpage}
\pagestyle{fancy}
\renewcommand{\headrulewidth}{0pt}
\renewcommand{\footrulewidth}{0.4pt}
\lfoot{{\scriptsize mis à jour le \today}}
\cfoot{ \vspace{-10pt} {\setFootFont \fontsize{8pt}{10pt} {\va Y}an{\vf n} {\va E}{\vb s}{\vb p}o{\vg s}{\vb i}{\va t}{\vd o}} }
\rfoot{\arabic{page}}

% HEADINGS
\usepackage{sectsty} 
\usepackage[normalem]{ulem} 
\sectionfont{\sffamily\mdseries\large\underline} 
\subsectionfont{\rmfamily\mdseries\scshape\normalsize} 
\subsubsectionfont{\rmfamily\bfseries\upshape\normalsize} 

% PDF SETUP
% ---- FILL IN HERE THE DOC TITLE AND AUTHOR
\usepackage[dvipdfm, bookmarks, colorlinks, breaklinks, pdftitle={Yann Esposito - CV},pdfauthor={Yann Esposito}]{hyperref}  

% Mes couleurs persos (je n'utilise plus vraiment l'orange en fait)
% Il faudrait utiliser #600 #d00 pour les rouges, 
% #ddd et #666 pour les gris.
\usepackage{color}
\definecolor{orange}{rgb}{1,0.5,0}
\definecolor{darkorange}{rgb}{0.5,0.25,0}
\definecolor{red}{rgb}{0.6875,0,0}
\definecolor{darkred}{rgb}{0.375,0,0}
\hypersetup{linkcolor=red,citecolor=red,filecolor=black,urlcolor=red} 

