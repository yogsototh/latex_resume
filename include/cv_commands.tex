% Yann Esposito
%
% Les commanes LaTeX pour les CV
%
% Des macros pour afficher correctement les contacts
% Les publications (en mode 'full' ou 'short')
%


% années dans la marge
\newcommand{\years}[1]{\marginpar{\scriptsize #1}}

% ======
% AUTEUR
% ======
%
% Simple : \auteur{François Dupond}
% Expert : \auteur{\va{F}ran\vb{ç}ois \vg{D}upond}
%
% les macros \va, \vb, ... \vg
% permette d'utiliser les variante des fontes
%
% Tout d'abord on sélectionne la fonte par défaut
% pour l'auteur ici Zapfino (très jolie script)
\newcommand{\setAuthorFont}[1]{
\color{darkred} 
\fontspec[
    Ligatures={Common,Rare},
    Variant=3
    ]
    {Zapfino}\fontsize{22pt}{25pt}\selectfont #1}%
\newcommand{\setFootFont}[1]{
\color{darkred} 
\fontspec[
    Ligatures={Common,Rare},
    Variant=3
    ]
    {Zapfino}\fontsize{10pt}{15pt}\selectfont #1}%
% Zapfino possède de multiples variantes pour ses lettres
% C'est pourquoi on en profite, une commande pour chaque
% Variante. Pour modifier une lettre dans son nom
% il suffit d'écrire \auteur{\vc{D}upond} par exemple
\newcommand{\va}[1]{\fontspec[Variant=1]{Zapfino}#1}
\newcommand{\vb}[1]{\fontspec[Variant=2]{Zapfino}#1}
\newcommand{\vc}[1]{\fontspec[Variant=3]{Zapfino}#1}
\newcommand{\vd}[1]{\fontspec[Variant=4]{Zapfino}#1}
\newcommand{\ve}[1]{\fontspec[Variant=5]{Zapfino}#1}
\newcommand{\vf}[1]{\fontspec[Variant=6]{Zapfino}#1}
\newcommand{\vg}[1]{\fontspec[Variant=7]{Zapfino}#1}
% La macro d'ateur proprement dite
\newcommand{\auteur}[1]{{\hspace{-2em}\vspace{-1em} \setAuthorFont{#1}}}

% =====
% TITRE
% =====
%
% Simple : \titre{Chercheur en informatique}
%
\newcommand{\setTitreFont}[1]{
\fontspec[]{Optima}\fontsize{18pt}{23pt}\selectfont #1}%
% La macro d'ateur proprement dite
\newcommand{\titre}[1]{\begin{minipage}[t]{.5\linewidth}\vspace{.5em}\flushright{\setTitreFont{#1}}\end{minipage}\\[1cm]}

% =======
% CONTACT
% =======
%
% exemple :
% ---
%
% \rue{18 rue des Anciens}
% \codePostal{012345}
% \ville{Arkkam}
% \telephone{(+33)1 23 45 67 89}
% \email{lovecraft@cthulhu.com}
%
% \afficheContactInfos
% ---
%
% batiment, rue, complementAdresse, codePostal, ville, pays
% telephone, site, email, faxnum
\newcommand{\batiment}[1]{\newcommand{\batimentValue}{#1}} % batiment
\newcommand{\rue}[1]{\newcommand{\rueValue}{#1}} % rue
\newcommand{\complementAdresse}[1]{\newcommand{\complementAdresseValue}{#1}} % complement
\newcommand{\codePostal}[1]{\newcommand{\codePostalValue}{#1}} % code postal
\newcommand{\ville}[1]{\newcommand{\villeValue}{#1}} % ville
\newcommand{\pays}[1]{\newcommand{\paysValue}{#1}} % pays
\newcommand{\telephone}[1]{\newcommand{\telephoneValue}{#1}} % téléphone
\newcommand{\site}[1]{\newcommand{\siteValue}{#1}} % site
\newcommand{\email}[1]{\newcommand{\emailValue}{#1}} % email
\newcommand{\faxnum}[1]{\newcommand{\faxValue}{#1}} % fax

% affichage de l'adresse
\newcommand{\addressPrefix}{\Letter\hspace{.8em}}
\newcommand{\afficheAdresse}{
    \addressPrefix
    \ifdefined\batimentValue
        Bât. \batimentValue, 
    \else\fi
        \rueValue\\
    \ifdefined\complementAdresseValue
        \phantom{\addressPrefix}\complementAdresseValue \\
    \else\fi
        \phantom{\addressPrefix}\texttt{\codePostalValue}, \villeValue\ifdefined\paysValue
    \\
        \phantom{\addressPrefix}\paysValue
    \else\fi
    \\[.2cm]
}

% Affiche l'adresse ainsi que les informations de contacts
\newcommand{\afficheContactInfos}{
    \begin{minipage}[t]{0.50\linewidth}
        \afficheAdresse
    \end{minipage}
    \hfill
    \begin{minipage}[t]{0.39\linewidth}
        \begin{flushleft}
        \ifdefined\telephoneValue
            \Telefon\hspace{.8em}{\texttt \telephoneValue}
        \else\fi
        \ifdefined\faxValue
            \\
            \fax\hspace{.8em}{\texttt \faxValue}
        \else\fi
        \ifdefined\emailValue
            \\
            \MVAt\hspace{.8em}\href{mailto:\emailValue}{\emailValue}
        \else\fi
        \ifdefined\siteValue
            \\
            \Mundus\hspace{.8em}\href{\siteValue}{\siteValue}
        \else\fi
        \end{flushleft}
    \end{minipage}
}

% ============
% PUBLICATIONS
% ============
% Les macros a utiliser :
% \article et \journalentry qui prennent 5 paramètres :
%   année
%   titre
%   auteur(s)
%   conférence ou journal
%   détails
%
% Pour afficher les publications en mode short
% Il suffit de décommenter la ligne suivante
% \newcommand{\short}{Short publications}


% Commandes pour les publications
\newcommand{\articleTitle}[1]{\emph{#1}}
\newcommand{\conf}[1]{{\textcolor{red}{#1}}}
\newcommand{\journal}[1]{\textcolor{red}{#1}}
\newcommand{\details}[1]{{\footnotesize #1}}

% paramètres:
%   année
%   titre
%   auteur(s)
%   conférence ou journal
%   détails
\newcommand{\article}[5]{
    \ifdefined\short
        \shortarticle{#1}{#2}{#3}{#4}{#5}
    \else
        \longarticle{#1}{#2}{#3}{#4}{#5}
    \fi
}
\newcommand{\journalentry}[5]{
    \ifdefined\short
        \shortjournalentry{#1}{#2}{#3}{#4}{#5}
    \else
        \longjournalentry{#1}{#2}{#3}{#4}{#5}
    \fi
}

\newcommand{\longarticle}[5]{
    \noindent\years{#1}\begin{minipage}[t]{\linewidth}\articleTitle{#2},\\#3\\\conf{#4}, \details{#5}.\end{minipage}\smallskip
}

\newcommand{\longjournalentry}[5]{
    \noindent\years{#1}\begin{minipage}[t]{\linewidth}\articleTitle{#2},\\#3\\\journal{#4}, \details{#5}.\end{minipage}\smallskip
}

\newcommand{\shortarticle}[5]{
    [#4]
}
\newcommand{\shortjournalentry}[5]{
    [#4~#1]
}

% ============
% PRESENTATION
% ============
% une macro \deuxcolones qui permet de simuler
% la presentation un peu comme avec un tableau
%
%   \deuxcolones{Total}{370 heures}
%
% on peut changer la taille des colones avec :
% \renewcommand{\dcleftwidth}{.15\linewidth}
% \renewcommand{\dcrightwidth}{.85\linewidth}
%
\newcommand{\dcleftwidth}{.25\linewidth}
\newcommand{\dcrightwidth}{.75\linewidth}

\newcommand{\deuxcolones}[2]{
\begin{minipage}[t]{\dcleftwidth}
#1
\end{minipage}
\begin{minipage}[t]{\dcrightwidth}
\small
#2
\end{minipage}
\medskip
}
